
With the continuous development of smart home infrastructure, smart home has gradually entered a new period characterized by smart services. A large number of complex and heterogeneous intelligent devices cooperate with each other to form a massive, intelligent and integrated context-aware intelligent home service. By analyzing the environment, state and even emotional information of the user, the device can make corresponding preparations with foresight. For example, it can provide accurate temperature control, air purification and other services according to the changes of the situation of the service object. It is a typical representative of intelligent home application. At present, smart home services tend to adopt end-user Programming. End-user Programming can reduce the learning process of Programming language for non-professionals, enabling non-professional users to create, modify or extend components at certain points in the system. Users can customize personalized intelligent home scene services according to their unique needs. To support end-user Programming, you need an infrastructure to control and coordinate devices. However, developing such an infrastructure in a smart home environment presents a number of technical challenges:

$(1)$Most devices do not expose their operating interfaces, making it impossible for end users to complete their programming. In addition, there are differences in brand and function of devices, providing different ways of data reading and function calling, which brings great complexity to device interaction and collaboration.

$(2)$The service needs to meet the needs of personalized scenarios. The differences in devices, types of services and spatial attributes of the scenarios make the interaction between services flexible, and at the same time bring great difficulties to the code logic writing of integrating these services.

In addition, end-user Programming needs to be able to effectively interpret user instructions. Knowledge graph is used to describe the concepts, entities, events and their relationships in the objective world. It can be used to describe the situational knowledge of the smart home, act as a bridge between system requirements and system implementation, and map users' instructions described in natural language to specific devices and services. Two challenges remain:	

$(1)$Knowledge graph is difficult to represent changes in smart home situations. Knowledge graph can organize structured data and represent knowledge with entities and relationships. However, the representation of situational knowledge requires knowledge graph to be able to perceive real-time scene information, while existing knowledge graph technologies are difficult to reflect real-time changes of data.

$(2)$Oriented to personalized service needs, it is necessary to conduct accurate reasoning on knowledge graph considering constraints of device type, location and other dimensions, so it is necessary for knowledge graph to have broader knowledge expression ability. In addition, how to transform the user's natural language instruction into the service capability of the device in the knowledge graph is also an urgent problem to be solved.

This paper proposes a Supporting End-user Programming towards Smart Home Services - A Method based on Runtime Knowledge Graph to solve the problems caused by system complexity in the mapping process of intelligent home service requirements to implementation. The main contributions include:

Firstly, a runtime knowledge graph for smart home is proposed, which represents situational knowledge of users, devices and locations through runtime concepts and relationship examples, so as to realize device interaction and collaboration at the model layer.

Secondly, a natural language-oriented service modeling and execution technology is proposed to generate intelligent home use case scenarios based on the runtime knowledge graph and map the natural language into executable programs on the runtime knowledge graph.

The above method is applied to the actual scenario of smart home, including XX functions on XX devices and XX smart home services. The experimental results show that the user can correctly formulate \% of the smart service at one time, and can formulate \% of the service after feedback.
